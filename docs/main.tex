\documentclass[a4paper,14pt]{extreport}
\usepackage[utf8]{inputenc}
\usepackage[english, russian]{babel}
\usepackage{listings}
\usepackage{graphicx}
\usepackage{float}
\graphicspath{{imgs/}}
\usepackage{amsmath,amsfonts,amssymb,amsthm,mathtools} 
\usepackage{pgfplots}
\usepackage{filecontents}
\usepackage{indentfirst}
\usepackage{eucal}
\usepackage{enumitem}
\frenchspacing

\usepackage{indentfirst} % Красная строка

\usetikzlibrary{datavisualization}
\usetikzlibrary{datavisualization.formats.functions}

\usepackage{amsmath}
\usepackage{fixltx2e}
\usepackage{caption}


\definecolor{bluekeywords}{rgb}{0,0,1}
\definecolor{greencomments}{rgb}{0,0.5,0}
\definecolor{redstrings}{rgb}{0.64,0.08,0.08}
\definecolor{xmlcomments}{rgb}{0.5,0.5,0.5}
\definecolor{types}{rgb}{0.17,0.57,0.68}

\usepackage{listings}
\lstset{language=[Sharp]C,
	captionpos=t,
	numbers=left, %Nummerierung
	numberstyle=\small, % kleine Zeilennummern
	frame=single, % Oberhalb und unterhalb des Listings ist eine Linie
	stepnumber=1,                   
	numbersep=5pt,                
	showspaces=false,
	tabsize=2,
	showtabs=false,
	breaklines=true,
	showstringspaces=false,
	breakatwhitespace=true,
	escapeinside={(*@}{@*)},
	commentstyle=\color{greencomments},
	morekeywords={partial, var, value, get, set},
	keywordstyle=\color{bluekeywords},
	stringstyle=\color{redstrings},
	basicstyle=\ttfamily\small,
}

\usepackage[left=3cm,right=1cm, top=2cm,bottom=2cm,bindingoffset=0cm]{geometry}
% Для измененных титулов глав:
\usepackage{titlesec, blindtext, color} % подключаем нужные пакеты
\definecolor{gray75}{gray}{0.75} % определяем цвет
\newcommand{\hsp}{\hspace{20pt}} % длина линии в 20pt
% titleformat определяет стиль
\titleformat{\chapter}[hang]{\Huge\bfseries}{\thechapter\hsp\textcolor{gray75}{|}\hsp}{0pt}{\Huge\bfseries}

\usepackage{array}
\newcommand{\head}[2]{\multicolumn{1}{>{\centering\arraybackslash}p{#1}}{#2}}



\begin{document}
	
\renewcommand{\contentsname}{Содержание}
\tableofcontents
\setcounter{page}{3}


\chapter*{Введение}
\addcontentsline{toc}{chapter}{Введение}

Методы прогнозирования многообразны, как и объекты, прогнозированием которых занимается человек. Ему всегда было интересно узнать будущее -- свое, своих близких, государства, экономики, предугадать погоду, определить как изменятся в ближайшее время курсы валют и т.п.

Но прогнозировать ситуацию важно не только из-за того, что это интересно, но и потому, что от предвидения будущего зависят действия человека. Если прогнозируется скачок цен на какой-то товар, то можно заранее закупить его. Фактически, можно сказать, что жизнь современного человека невозможно себе представить без прогнозирования

Прогноз -- это результат индуктивного вывода, когда по характеру ограниченного множества значений показателей или взаимосвязи факторов делается вывод о том, что и остальные, еще не наблюдаемые значения этих показателей или взаимосвязи будут обладать аналогичными свойствами \cite{hse_pred}.
	
\chapter{Аналитическая часть}



\chapter{Конструкторская часть}



\chapter{Технологическая часть}



\chapter*{Заключение}
\addcontentsline{toc}{chapter}{Заключение}


\chapter*{Список литературы}
\addcontentsline{toc}{chapter}{Список литературы}

\begin{thebibliography}{3}
	\bibitem{hse_pred} Методы и модели социально-экономического прогнозирования : учебник и практикум для академического бакалавриата. В 2-х т. Т. 1. Теория и методология прогнозирования / И. С. Светуньков, С. Г. Светуньков. — М. : Издательство Юрайт, 2014. — 351 с. — Серия : Бакалавр. Академический курс.
	
\end{thebibliography}
	
\end{document}